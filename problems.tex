\documentclass[10pt]{exam}

\usepackage{graphicx}
\usepackage{enumitem}
\usepackage{fullpage}
\usepackage{amsmath}
\usepackage{amsthm}
\usepackage{amssymb}
% \usepackage{url}
% \usepackage{hyperref}



\theoremstyle{plain}
\newtheorem{theorem}{Theorem}
\newtheorem{corollary}[theorem]{Corollary}
\newtheorem{conjecture}[theorem]{Conjecture}
\newtheorem{proposition}[theorem]{Proposition}
\newtheorem{hypothesis}[theorem]{Hypothesis}
\newtheorem{assumption}[theorem]{Assumption}
\newtheorem{sublemma}[theorem]{Sublemma}
% \newtheorem{question}[theorem]{Question}


\theoremstyle{definition}
\newtheorem{definition}[theorem]{Definition}
\newtheorem{remark}[theorem]{Remark}
\newtheorem{remarks}[theorem]{Remarks}
\newtheorem{example}[theorem]{Example}
\newtheorem{exercise}[theorem]{}
\newtheorem{notation}[theorem]{Notation}


\newcommand{\RR}{\mathbb{R}}
\newcommand{\ZZ}{\mathbb{Z}}
\newcommand{\QQ}{\mathbb{Q}}
\newcommand{\CC}{\mathbb{C}}
\newcommand{\NN}{\mathbb{N}}
\newcommand{\qand}{\quad\text{and}\quad}
\newcommand{\qqand}{\qquad\text{and}\qquad}

\newcommand{\lra}{\longrightarrow}

\renewcommand{\phi}{\varphi}
\renewcommand{\epsilon}{\varepsilon}

\DeclareMathOperator{\rank}{rank}
\DeclareMathOperator{\tip}{tip}
\DeclareMathOperator{\tail}{tail}
\DeclareMathOperator{\dist}{dist}
\DeclareMathOperator{\rref}{rref}
\DeclareMathOperator{\adj}{adj}
\DeclareMathOperator{\var}{var}
\DeclareMathOperator{\cov}{cov}
\DeclareMathOperator{\pr}{proj}


\renewcommand{\phi}{\varphi}
\newcommand{\mat}[1]{\begin{bmatrix}#1\end{bmatrix}}
\newcommand{\Ab}{{}[A\mid b]}
\newcommand{\Azero}{{}[A\mid 0]}

\newcommand{\COS}[1]{\cos\big(2\pi k (t+#1)\big)}
\newcommand{\SIN}[1]{\sin\big(2\pi k (t+#1)\big)}
\newcommand{\floor}[1]{\left\lfloor#1\right\rfloor}


\newcommand{\bbeta}{\boldsymbol{\beta}}
\newcommand{\ba}{\mathbf{a}}
\newcommand{\bb}{\mathbf{b}}
\newcommand{\bp}{\mathbf{p}}
\newcommand{\bq}{\mathbf{q}}
\newcommand{\br}{\mathbf{r}}
\newcommand{\bu}{\mathbf{u}}
\newcommand{\bv}{\mathbf{v}}
\newcommand{\bw}{\mathbf{w}}
\newcommand{\bx}{\boldsymbol{x}}
\newcommand{\by}{\boldsymbol{y}}
\newcommand{\bz}{\mathbf{z}}
\newcommand{\bzero}{\mathbf{0}}

\newcommand{\cO}{\mathcal{O}}
\newcommand{\fa}{\mathfrak{a}}
\newcommand{\fb}{\mathfrak{b}}
\newcommand{\fp}{\mathfrak{p}}
\newcommand{\fq}{\mathfrak{q}}

\DeclareMathOperator{\spn}{span}
\DeclareMathOperator{\disc}{disc}
\DeclareMathOperator{\Hom}{Hom}

\renewcommand{\hat}{\widehat}

\newcommand{\fP}{\mathfrak{P}}

\DeclareMathOperator{\Gal}{Gal}
\DeclareMathOperator{\Fr}{Fr}

\begin{document}

\title{Math 311 -- Algebraic number theory -- Practice problemd}
\author{Instructor: Matthew Greenberg}
\maketitle

\begin{questions}


\question
Let $A$ be an abelian group. Define the \emph{rank of $A$} to be 
\[
\sup\{|X| : \text{$X\subseteq A$ is $\ZZ$-linearly independent}\}.
\]
\begin{parts}
\part
Define what it means for an abelian group to be:
\begin{itemize}
\item free,
\item finitely generated,
\item torsion-free.
\end{itemize}

\part
State the structure theorem for finitely-generated abelian groups. Use it to prove that an abelian group is free of rank $r<\infty$ if and only if $A\approx \ZZ^n$.

\part
State the \emph{elementary divisors theorem}.
Use it to prove that if $A$ and $B$ are abelian groups such that if $B$ is free of rank $n<\infty$ and $A\subseteq B$, then $A$ is free of rank $m\leq n$ with equality if and only if $B/A$ is finite.
Further, show that if $m=n$ then $[B:A]$ is the product of the elementary divisors of $A$ with respect to $B$.
\end{parts}

\question Let $K$ be a number field of degree $n$ with ring of integers $\cO_K$.

\begin{parts}
\part Prove that $\cO_K$ is a free abelian group of rank at most $n$.
\[
\rank \cO_K\leq \dim K.
\]
(Hint: Explain why (i) $\cO_K$ is torsion-free and (ii) a set of $>n$ elements of $K$ are $\ZZ$-linearly dependent.)

\part Let $a\in K$. Prove that there is an integer $n>0$ such that $na\in \cO_K$. 

\part Prove that $\cO_K$ contains a $\QQ$-basis of $K$.
(Use the preceding exercise.) Deduce that
\[
\rank \cO_K\geq n.
\]

\end{parts}


\question
Let $V$ be an $n$-dimensional $\QQ$-vector space and let $L$ be an abelian subgroup of $V$.

\begin{parts}
\part
Prove that if $L$ is free of rank $\leq n$ if and only if $L$ is finitely generated.
(Use the structure theorem for finitely generated abelian groups.)

\part Prove that a subset $X$ of $V$ is $\QQ$-linearly independent if and only if it is $\ZZ$-linearly independent.

\part Prove that
\[
\sup\{|X| : \text{$X$ is a linearly independent subset of $L$}\}\leq n.
\]
(We are \emph{not} assuming $L$ is finitely generated.)

\part Give an example of an abelian subgroup of $V$ that is not finitely generated.
\end{parts}

\question
Let $V$ be an $F$-vector space and let
\[
B:V\times V\lra F
\]
be an $F$-bilinear form.
\begin{parts}
\part Prove that the following conditions are equivalent:
\begin{enumerate}
\item $B(x,V)=0$ if and only if $x=0$.
\item The \emph{duality map}
\[
\delta:V\lra V^*:=\Hom_F(V,F)
\]
defined by
\[
\delta(x)(y)=B(x,y)
\]
is injective.
\end{enumerate}
If $B$ satisfies these conditions, it is called \emph{nondegenerate}.

\part
Suppose $B$ is nondegenerate and $V$ is finite-dimensional. Prove that $\delta$ is an isomorphism. (Hint: What is the dimension of $V^*$?)

\part
Suppose that $B$ is nondegenerate and $\bv=(v_1,\ldots,v_n)$ is an $F$-basis of $V$.
Prove that there are is a unique \emph{$B$-dual basis} $\bv^B=(v_1^B,\ldots,v_n^B)$ of $V$ such that
\[
B(v_i,w_j)=\begin{cases}1&\text{if $i=j$,}\\0&\text{if $i\neq j$.}\end{cases}
\]
(Hint: Since $\bv$ is an $F$-basis of $V$, there are unique $F$-linear functionals
\[
\ell_j:V\lra F
\]
such that
\[
\ell_j(v_i) = \begin{cases}1&\text{if $i=j$},\\0&\text{if $i\neq j$.}\end{cases}
\]
Now use the injectivity of $\delta$.)


\part
Suppose that $\dim V=n$, that $B$ is nondegenerate and that $\bv$ is an $F$-basis of $V$.
Let $[B]_\bv\in F^{n\times n}$ be the matrix whose $(i,j)$-component is $B(v_i,v_j)$.
Show that
\[
B(x,y) = [x]_\bv^t[B]_\bv[y_\bv],
\]
where $[x]_\bv, [y]_\bv\in F^{n\times1}$ are the coordinate vectors of $x$ and $y$ with respect to $\bv$, respectively.

\part Suppose that $\dim V=n$, that $B$ is nondegenerate, and that $\bv$ and $\bv'$ are $F$-bases of $V$. Let $A\in F^{n\times n}$ be the matrix mapping the $\bv$-coordinates of $x\in V$ to its $\bv'$ -coordinates:
\[
A[x]_\bv=[x]_\bv'.
\]
Prove that
\[
A^t[B]_{\bv'}A = [B]_\bv.
\]

\part
Set
\[
\disc\bv=\det [B]_\bv,
\]
Prove that
\[
\disc\bv \equiv \disc \bv'\pmod{F^{\times 2}},
\]
where $\bv'$ is another $F$-basis of $V$ and $F^{\times2}=\{x^2 : x\in F^\times\}$. (Hint: Use the preceding exercise and properties of the determinant.)

The \emph{discriminant of $B$}, denoted $\disc B$, is defined to be the image of $\disc \bv$ in $F^\times/F^{\times 2}$. By the above,
\[
\disc B\in F^\times/F^{\times2}
\]
does not depend on our choice of $\bv$.
\end{parts}

\question
Suppose that $V$ is an $n$-dimensional $\QQ$-vector space equipped with a nondegenerate, symmetric, $\QQ$-bilinear form
\[
B:V\times V\lra \QQ
\]
Let $L$ be an abelian subgroup of $V$.

\begin{parts}
\part
Suppose that $\bv=(v_1,\ldots,v_n)$ and $\bv'=(v_1',\ldots,v_n')$ are $\ZZ$-bases of $L$. Prove that
\[
\disc \bv = \disc \bv'.
\]
Define $\disc(L,B)$ to be this common value.

\part Suppose that $M$ is an abelian subgroup of $V$ with $L\subseteq M$. Show that $M$ is $B$-integral and that
\[
\disc(M,B) = [L:M]^2\disc(L,B).
\]
(Hint: Use the elementary divisors theorem.)
Conclude that if $\disc(L,B)$ is squarefree then $L$ is a maximal $B$-integral subgroup of $V$.

\part 
Define the \emph{$T$-dual subgroup}
\[
L^T = \{x\in V : T(x,L)\subseteq\ZZ\}.
\]
Prove that $L_1\subseteq L_2$ implies $L_2^T\subseteq L_1^T$.

\part Suppose that $\bv=(v_1,\ldots,v_n)$ is a basis of $L$. Prove that $\bv^T=(v_1^T,\ldots,v_n^T)$ is a basis of $L^T$.

\part
Prove that
\[
|\disc(L, B)| = [L^T:L].
\]
\end{parts}

\question
\begin{parts}
\part Define the \emph{trace map}
\[
t:K\lra \QQ.
\]
\part Prove that $t$ is $\QQ$-linear.

\part
Define the \emph{trace form}
\[
T:K\times K\lra \QQ.
\]
\part
Prove that $T$ is $\QQ$-bilinear. (Use the fact that $t$ is $\QQ$-linear.)

\part
We proved in class that $T$ is nondegenerate. Remind yourself why this is.

\part
Prove that $\cO_K\subset\cO_K^T$, i.e., that
\[
T(\cO_K,\cO_K)\subseteq\ZZ.
\]

\part Let $L$ be a subgroup of $\cO_K$. Prove that $\cO_K^T\subseteq L^T\subset$.

\part Let $\bv=(v_1,\ldots,v_n)$ be a basis of $K$ with $v_i\in \cO_K$ (such a basis exists by a previous exercise) and let
\[
L=\ZZ v_1+\cdots+\ZZ v_n.
\]
Then $L^T$ free abelian group of rank $n$ containing $\cO_K$. (Why?) Deduce that $\rank \cO_K\leq n$. Combine this with the result of a previous exercise to deduce that $\rank \cO_K=n$.

\part
Let $\fa$ be a nonzero ideal of $\cO_K$.
Prove that $\fa$ contains a positive integer $n$.
It follows that $n\cO_K\subseteq \fa$. Deduce that $\rank \fa=n$.

\part Let $\bv=(v_1,\ldots,v_n)$ be an integral basis of $K$, i.e., $\ZZ$-basis of $\cO_K$. Define the \emph{discriminant $d_K$ of $K$} by
\[
d_K = \disc(\cO_K,T).
\]
Explain why $d_K$ is well-defined, i.e., is independent of our choice of basis.

\part Let $\cO$ is a subring of $K\cap\bar{\ZZ}$ with $\rank\cO=n$. 
Suppose that $\disc(\cO, T)$ is squarefree. Show that $\cO=\cO_K$ and that $d_K=\disc(\cO, T)$.

\part Let $\cO$ be a subring of $K\cap \bar{\ZZ}$ with $\rank \cO=n$. Suppose that $\disc(\cO,T)=p^2$, where $p$ is a prime number. Show that $\cO=\cO_K$ and that $d_K=\disc(\cO, T)$.
\end{parts}

\question
Let $a$ be a root of $f(x)$ and let $K=\QQ(a)$. Compute:
\begin{itemize}
\item $\disc(1,a,\ldots,a^{\deg f-1})$
\item $r_1(K)$, $r_2(K)$, $\rank\cO_K^\times$
\end{itemize}
In the cases where $\rank\cO_K^\times\geq 1$, can you write down any units of infinite order? How about a linearly independent subset of $\cO_K^\times$ of maximal rank?)

\medskip
\begin{parts}
\part
$f(x)=x^2-m$

\part
$f(x) = x^2 -x + \dfrac{1-m}4$,\quad $m\in\ZZ$, $m\equiv 1\pmod 4$.

\part
$f(x) = x^3+x^2-1$

\part
$f(x)=x^3 + x^2 - 2x -  1$

\part
$f(x)=$ minimal polynomial of $\zeta_7+\zeta_7^{-1}$, where $\zeta_7=e^{2\pi i/7}$

\part
$f(x)=$ minimal polynomial of $\dfrac1{\sqrt2}(1+\sqrt{-1})$

\part
$f(x)=$ minimal polynomial of $\sqrt{5}+\sqrt{-1}$
\end{parts}

\question Let $K=\QQ(\sqrt{-5})$.

\begin{parts}
\part Prove that $2$ ramifies in $K$.

\part Let $\fp_2$ be the unique (prime) ideal of $\cO_K$ such that $\fp^2=2\cO_K$. Prove that $\fp_2$ is not a principal ideal. Deduce that the class number of $K$ is at least $2$.
\part By computing the Minkowski bound for $K$, show that $K$ has class number $2$.

\part Prove that $\fp_2=(2,1+\sqrt{-5})$. (Hint: It suffices to show (why?) that $(2,1+\sqrt{-5})^2=2\cO_K$.)

\part Prove that $3$ splits in $\cO_K$; write $3\cO_K=\fp_3\bar{\fp}_3$. Show that $\fp_3$ and $\bar{\fp}_3$ is not principal.

\part Explain why $\fp_2\fp_3$ and $\fp_2\bar{\fp}_3$ are a principal ideals. Identify generators. (You don't need to a presentation of $\fp_3$ to do this.)
\end{parts}


\question
Let $f(x) = x^3-ax^2-(a+3)x-1$.
\begin{parts}
\part Prove that $f(x)$ is irreducible.

\part Let $\rho=\rho_1$ be a root of $f(x)$ and let $K=\QQ(\rho)$.
Verify that
\[
\rho_2:= \frac{-1}{1+\rho_1}\quad\text{and}\quad
\rho_3:= \frac{-1}{1+\rho_3}
\]
are the other roots of $f(x)$:
\[
f(x) = (x-\rho_1)(x-\rho_2)(x-\rho_2).
\]
Deduce that $K$ is totally real: $$r_1(K)=3.$$

\part Show that the $\rho_j$ are units:
\[
\rho_j\in\cO_K^\times,\quad j=1,2,3.
\]

\part
Prove that
\[
\disc(1,\rho,\rho^2)=(a^2+3a+9)^2.
\]


\part
Suppose $$p:= a^2+3a+9$$ is prime.
For example, $a=-1$, $1$, and $2$ give $p=7$, $13$, and $19$, respectively.
Prove that $\cO_K=\ZZ[\rho]$.

\part $(**)$
Show that $p\equiv 1\pmod 3$, making $\frac{p-1}3$ an integer. Let $q\neq p$ be another prime. Prove that
\[
q^{\frac{p-1}3}\equiv\begin{cases}1&\text{if $q$ splits of $K$,}\\-1&\text{if $q$ is inert in $K$.}\end{cases}\pmod{p}.
\]
In other words, $q\neq p$ splits in $K$ if and only if $q$ is a cube modulo $p$.
\end{parts}



\question
Let $K$ be a Galois number fields. Let $p\in\ZZ$ be a prime and let $\fp$ be a prime ideal of $\cO_K$ such that $\fp\mid p$. Assume that $p$ is unramified in $K$.
\begin{parts}
\part Define the \emph{decomposition group $D_{\fp|p}$}.
\part Define the \emph{Frobenius automorphism} $\Fr_{\fp|p}\in D_{\fp|p}$.
\part Let $K$ be a quadratic field and let $d_K$ be its discriminant. Let $p$ be a prime with $p\nmid d_K$. Identify the image of $\Fr_{\fp|p}$ under the natural (unique) map
\[
\Gal(K/\QQ)\lra \{\pm1\}.
\]
You don't need to justify your answer for this part.
\end{parts}

\question
Let $K$ and $L$ be a Galois number fields with $K\subset L$. Let $p\in\ZZ$ be a prime and let $\fp$ be a prime ideal of $\cO_K$ such that $\fp\mid p$, and let $\fP$ be a prime ideal of $\cO_L$ such that $\fP\mid\fp$. Assume that $p$ is unramified in $L$ and, hence, in $K$.

\medskip
Show that the natural map
\[
\Gal(L/\QQ)\lra \Gal(K/\QQ)
\]
maps $D_{\fP|p}$ into $D_{\fp|p}$ and $\Fr_{\fP|p}$ onto $\Fr_{\fp|p}$.



\question
Let
\[
K=\QQ(\sqrt5,\sqrt{-1}).
\]
\begin{parts}
\part For a prime $p\in\ZZ$, define $e$, $f$, and $g$ by
\[
p\cO_K=(\fP_1\cdots\fP_g)^e,\quad f=[k(\fP_i)/k(p)].
\]
Here, $k(\fP_i)$ and $k(p)$ are the residue class fields of $\fP_i$ and $p$, respectively.

\medskip
Enumerate all possible triples $(e,f,g)$. For each triple $(e,f,g)$, provide a corresponding $p$.

\part There is a third quadratic subfield of $K$; find it. Call it $E$.

\part 
Determine the triples $(e',f',g')$ and $(e'',f'',g'')$ characterized by
\begin{align*}
2\cO_E &= (\fp_1\cdots\fp_{g'})^{e'}& f'&=[k(\fp_i)/k(p)],\\
\fp_i\cO_K &= (\fP_1\cdots\fP_{g''})^{e''}& f''&=[k(\fP_j)/k(\fp_i)].
\end{align*}
\end{parts}


\end{questions}

\end{document}